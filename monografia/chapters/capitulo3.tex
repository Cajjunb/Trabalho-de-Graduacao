\chapter{Fundamentos Básicos para a Produção da Voz}
	Neste capitulo serão introduzidos princípios físicos e biológicos da produção da voz. Além dos conceitos introduzidos aqui, serão analisados os processos mecânicos da geração da voz por parte da laringe em conjunto com o trato vocal pois são de extrema importância para a concepção dos modelos computacionais e matemáticos utilizados para se desenvolver o sintetizador.

	\subsection{A Biomecanica da Laringe}
	
	Primeiramente, devemos ter em mente que o principal papel da laringe não é a produção de voz e sim a proteção das vias respiratórias. Dito isto, podemos fazer uma simples analise, visto que apesar de seu papel principal, a laringe também atua como um instrumento da fala humana. Se analisarmos os instrumentos criados pelo homem podemos notar que estes dependem basicamente de sua geometria, do material que o compõe e da interação de suas partes acústicas. Do mesmo modo, a laringe possui uma determinada geometria, é composta por tecido humano e em conjunto como trato vocal compõe a parte acústica do nosso corpo. Entretanto, nada é tão simples, a sua geometria e as propriedades do material humano envolvidos na produção do som são bastante irregulares~\cite{IngoTitze}. Outra analogia interessante é sobre o instrumento e quem o utiliza. Um bom pianista por exemplo, sua musica é boa porque ele é habilidoso como instrumento? Ou sua musica é boa por que o instrumento é bem feito e o som gerado por este é agradável? Ou os dois? Essas perguntas também podem ser feitas com respeito a voz. Para entendermos o que influencia na qualidade da síntese da voz é necessário analisar a biomecânica da voz, que nada mais é analisar o movimento do material vivo e as forças atuantes sobre ele~\cite{IngoTitze}.
	
	\subsection{Fatores Biológicos que Afetam a Produção de Som na Laringe}
	A parte da (bio)mecânica que se relaciona diretamente com a atuação da laringe na produção do som é a mecânica dos meios contínuos, que é a parte da mecânica que lida com a matéria distribuída sobre uma determinada região no espaço, e consequentemente, se contrapõe à mecânica de partículas. 
	
	Dentro da mecânica de meios contínuos, mais especificamente, a parte que irá nos auxiliar no estudo do comportamento da laringe se chama mecânica de sólidos e fluidos. Dito isto, analisaremos a seguir alguns conceitos físicos que tem forte ligação com os processos que ocorrem na região da laringe durante a produção do som: 
	
	\subsubsection{Tensão e Deformação}
	 São características de forças atuantes em superfícies, como por exemplo a resistência do ar. 

	\subsubsection{Tensão }
	 é quantidade de força por unidade de área~\cite{IngoTitze}, podemos escrever na forma da equação~\ref{eq:1} :
	 
	 \begin{equation}
	 	\centering
	 	\[
		 	\sigma = \frac{f}{A}
	 	\]
	 	\label{eq:1}
	 \end{equation}
	Onde σ é a tensão, f é a força aplicada e A a área de aplicação desta força. 
	
	\subsubsection{Deformação  }
	é a medida de deformação de um meio após a aplicação de uma tensão~\cite{IngoTitze} e pode ser escrito na forma da equação  :
	
	\begin{equation}
	\centering
		\[
		\epsilon = L - \frac{L_0}{L_0}
		\]
		\label{eq:2}
	\end{equation}
	
	Onde  é a medida de deformação, L é o comprimento após a tensão e L0 é o comprimento antes da tensão.
	
	Normalmente uma dada deformação em uma dimensão resulta em uma deformação oposta em outra dimensão em um dado meio. Se uma deformação é uniforme por todo o corpo de um objeto, então chamamos de compressão, se o volume diminui por conta desta deformação, e expansão, se o volume aumenta. 

	\subsubsection{Viscosidade}
	É a velocidade de deformação(consequentemente, de restauração) de um determinado fluido quando atuam forças de tensão no mesmo. Matematicamente pode ser expresso conforme a equação~\ref{eq:3}:
	
	\begin{equation}
		\centering
		\[
		\epsilon = \eta \frac{d_e}{d_t}
		\]
		\label{eq:3}
	\end{equation}

	Para η viscosidade e t tempo. Quanto maior a viscosidade, mais devagar será a deformação de um meio. 
	
	\subsection{Modelo Massa-Mola Auto Sustentável}:
	A fechadura e abertura da glote num sistema massa mola de apenas um lado

	\[
		P = (1 - \frac{a2}{a1})*(Ps- Pi) + Pi
	\]
	
	\subsection{Elasticidade}
	 É uma propriedade do meio que determina quão completa será a restauração do meio após uma dada deformação.
	 
	Os conceitos e propriedades descritos acima são extremamente importantes para se entender a manutenção da produção do som. As pregas vocais são músculos e músculos são compostos por fibras, logo, as pregas vocais consistem de uma grande concentração de fibras. Além disso, entre as fibras que compõe as pregas vocais existem também fluidos atuantes, o que caracteriza as pregas vocais como um material viscoelástico. Para se entender a capacidade de absorção e regeneração das pregas vocais, em detrimento das vibrações de alta frequência e as pressões do ar, deve-se primeiramente estudar as propriedades absorcivas do material que as compõe.
	
	Ou seja, em outras palavras, deve-se estudar as propriedades mecânicas do tecido viscoelástico, e uma ferramenta que facilita o entendimento é o estudo da curva força-alongamento de um material. Entretanto, construir uma curva de força-alongamento depende essencialmente da geometria da amostra do material~\cite{IngoTitze} e, por se tratar de uma material biológico, é difícil obter uma geometria precisa pois as fibras estão constantemente se reorientando em detrimento de lesões e cortes fibrosos. Para viabilizar este estudo, Titze~\cite{IngoTitze} sugere normalizar as forças atuantes e as deformações resultantes para que não haja a dependência direta da geometria. Essa normalização se dá através da substituição da curva força-alongamento por um curva tensão-deformação. A Figura ~\ref{fig:curvaTensao} retirada do estudo feito por Titze~\cite{IngoTitze} demonstra uma curva hipotética de tensão-deformação para os tecidos que compõe as pregas vocais humanas. Esta Figura ilustra o comportamento das fibras das pregas vocais através da relação entre uma força atuante e a deformação gerada por esta. A importância desta analise se deve ao fato de que é possível estabelecer uma relação direta entre nódulos vocais e uma fonação prolongada, alta(em termos de frequência) e intensa. A partir da análise desta curva é então possível estabelecer um precedente para a formação de nódulos vocais: a frequência e amplitude da vibração estão diretamente ligadas ao surgimento de um nódulo vocal e consequentemente o de uma fenda pois a força de impacto entre as pregas vocais é proporcional à altura tonal quando acima do tom natural e à intensidade durante a fonação.
	
	
	\begin{figure}
		\includegraphics{figura1}
		\caption{ Curva Hipotética Tensão-Deformação das Cordas vocais Humanas~\cite{IngoTitze}}
		\label{fig:curvaTensao}
	\end{figure}
	
	
	\subsection{Synpath}
		O SynPath é um sintetizador computacional, desenvolvido em linguagem Python [13, 20, 28], criado por Lucero~\cite{LuceroZueiro1}. Este software é uma extensão do sintetizador concebido por Fraj[19], incorporando um modelo de vibração para as pregas vocais. O seu propósito é aumentar a fidelidade fisiológica do sintetizador de Fraj e permitir o controle direto dos sons sintetizados em termos de parametrização da laringe. Para se obter um simples controle sobre o sintetizador e facilitar o seu uso para aplicações práticas, é necessário que a representação das pregas vocais seja simples. Além disso, o modelo das pregas vocais deve garantir variações suaves no fluxo gerado na glote. A falta de suavidade gera timbres não naturais e consequentemente a perda da fidelidade fisiológica buscada. Sendo assim, o modelo multi-massa para representação das pregas vocais não pode ser utilizado visto que produz variações não suaves e é um modelo matematicamente muito complexo levando a instabilidades numéricas, afetando o uso para aplicações práticas. O SynPath tomou como base para a representação das pregas vocais o modelo de onda mucosa desenvolvido por Titze [26]. Basicamente, o modelo é um oscilador mono-massa, conforme descrito no Capitulo 3, incorporando a transferência de energia do fluxo de ar para as pregas vocais. Entretanto, o modelo de Titze possui duas restrições, uma que foi solucionada e aplicada no desenvolvimento do SynPath e a outra que ainda não foi solucionada e portanto é também uma restrição do modelo computacional do SynPath. A primeira restrição é que o modelo de Titze foi concebido para o estudo em pequenas oscilações, sendo que para oscilações de maior amplitude, este não é apropriado. Entretanto, esse modelo foi extendido por Lucero [11] para abranger maiores oscilações utilizando um mecanismo limitador de amplitude durante as oscilações [14]. Mesmo com essa extensão para maiores amplitudes, o modelo ainda apresentava uma outra restrição, um atraso pequeno para o deslocamento da onda no canal glotal. A consequência desse atraso é que a pressão limite para que ocorra a vibração se torna independente da frequência de vibração [15]. Porém, sabe-se que um esforço maior é necessário para que tons mais agudos sejam vocalizados, ou seja, a pressão para que se ocorra vibração em frequências maiores(tons maiores) é maior. Essa restrição ainda não foi solucionada tendo em vista a dificuldade de se realizar o supracitado computacionalmente. Mesmo com essa restrição, o modelo de Lucero [11] é o modelo utilizado para
		26
		a representação do caráter oscilatório das pregas vocais no software SynPath, sendo que esta restrição não solucionada não tem forte influência no produto final.
		
	
		\subsubsection{Requisitos Funcionais}
		
		O Synpath é consistido também dos seguintes requisitos funcionais, os requistos funcionais são as funcionalidades que o sistema executará\cite{SWEBOK}
		
		\begin{itemize}
			\item 1 - Validação do Parâmetros passados pelo Usuário, se condizem com restrições do programa.
			\item 2 - Plotar um gráfico inicial do trato vocal de acordo com os parâmetros do usuário.
			\item 3 - Plotar três gráficos referentes as propriedades da voz simuladas com os parâmetros fornecidos pelo usuário. – O primeiro gráfico refere-se às posições adotadas pelas cordas vocais, a área da glótis, ao fluxo de ar nessa área e às características desse fluxo. – O segundo gráfico refere-se às características do som gerado pela simulação física do aparato fonador pelo programa. – O terceiro gráfico refere-se ao espectro de frequência do som gerado e do fluxo da glótis.
			\item 4 - Gerar um arquivo de texo com as características de voz gerada, frequencia,amplitude, ruído da voz, entre outros
			\item 5 Gerar um arquivo de som de voz simulada.
		\end{itemize}
		
		\subsubsection{Requisito Não-Funcionais}
		
		O Synpath consiste também dos seguintes requisitos não funcionais, requisitos não funcionais são requisitos são parametros de qualidade, requisitos que limitam as funcionalidades do sistema\cite{SWEBOK}.
		
		\begin{itemize}
			\item 1 - O Sistema deve produzir os gráficos que os requisitos funcionais delimitaram em um intervalo de 1(um) minuto. 
			\item 2  - Após gerar os gráficos e os exibi-los o arquivo texto e o arquivo de audio deverão ser exibidos
			\item 3 - Para que o sistema esteja funcional é necessário ter instalado os pacotes: MatPlotLib e NumPY
			\item 4 - O sistema deve ser executado em plataformas de um sistema operacionais como Windows, Linux ou MacOs, Versões recentes de acordo com a data desse documento. 
			
		\end{itemize}
		
	
	
	
	
	\subsection{HMMs}
		Minera-se de várias partituras musicais para treino. Os dados minerados dessas músicas são fonemas, altura, intensidade e os intervalos, isto é relação com outras notas.Esses dados são convertidos e mapeados em "labels" dependentes de contexto~\cite{DegottexNada}.Após isso as HMM's são treinados através dos dados de treinamento usando o algoritmo EM.\cite{GudnasonNada}. Após isso ocorre a fase de sintesis, usa-se outra partitura para ser convertida em "labels" dependentes de contexto e estima-se quais "labels" preprocessadas são correspondentes.\cite{TakashiNose}
	
		\subsection{MHRSMM}
		Uma variação HSMM. Modelo de multipla regressão HSMM.
		Parametros importantes são μi e mi dos outputs pdfs
		
		\[ \mu_i = H_bi \xi \] 
		\[	m_i   = H_pi \xi  \]
		\[ \xi   = [1,v_1,v_2,...,v_L]^T \]	
		\[ \xi   = [1,v^T]^T	\]	
		
		Onde L é a dimensão do vetor de estilo e vi  é  a intensidade do  enésimo estilo de canto.
		
		Um exemplo de um vetor de estilos de canto de tamanho L =  2 e L = 3.
		
		\includegraphics{exemploHMM.png}
		
		\subsubsection{Controle do Sintentisador de voz cantada baseado em MRHSMM}
		
		\includegraphics{esquemaHMM.png}
		
		Durante a fase de síntese o usuário do programa adiciona vetores de estilos de acorodo com a intenção e a expressividade pretendida.
		Parametros de output como duração são gerados pelos vetores de estilos dados e matrizes de regressões treinadas usando MRHSMMs
		
		Resultado de todo esse processo é um sequência HSMM usando parametros de geração de fala
		
		MRHSMM possui uma dificuldade de gerar contorno F0 que acompanhe o contexto de mudança de altura das notas o author TAKASHI NOSE, propõe um treinamento de HSMM e HMM nos parametros 
		
	
	\subsection{FrameWorks Sintetisador de Voz}
	
		Frame Work de um Sistema Sintetisador de Voz:
		
		\includegraphics{frameWork.png}
		
		\subsubsection{Input}
		Consiste da partitura, letra e emoção.
		o input é analisado e derivado em uma transcrição fonética, alinhamento com a performance alvo ou dados contextuais.\cite{FrameWork}\linebreak
		
		\subsubsection{Expressão}
		Expressão músical é um conceito intuitivo porém dificil de se definir. A expressão é chave na percepção da qualidade e naturalidade musical.
		No caso da voz cantada implica-se usar vários outros parametro além de frequencia e amplitude. Psicológicamente contorno do timbre, vibrato, tremolo, timing fonético.\cite{FrameWork}\linebreak
	
	
	\subsection{Envoltoria F0}
	
	Envoltórias F0 são usadas para expressar informação linguistica, para-linguística e não-linguistica.\cite{SaitouF0}
	\linebreak
	
	As Envoltória F0 apresentam três (3) características importantes que fazem diferenciar uma voz falada a uma voz cantada.\cite{SaitouNada}
	\linebreak
	\begin{itemize}
		\item 1 - O alcance dinâmico de uma envoltória F0 é mais largo que o de uma voz falada
		\item 2 - A envoltória F0 corresponde e tende a se manter estável em uma nota. A mudança de nota de uma envoltória F0 corresponde a melodia da música
		\item 3 - Existem muita flutuações f0 que são apenas observadas em apenas vozes cantadas
	\end{itemize}
	
	
	\subsection{Sintese de Voz em Mandarím}
		Utiliza-se a técnica HNM para a sintese da voz cantada em mandarím. HNM significa , "harmonic plus noise model".
		O modelo HNM divide o espectro de um sinal em dois(2) com larguras não iguais para modelagem melhor do espectro.\cite{LinRobos}
	