
\chapter{Conceitos Básicos de Sintese de Voz}
\section{Anatomia da Voz}
	\subsection{Orgãos}
	Orgãos envolvidos com a sintese da voz:
	
	\begin{itemize}
		\item Diafragma
		\item Pulmões
		\item Traqueos
		\item Esôfago
		\item laringe
		\item epiglote
		\item faringe
		\item palato , lingua
		cavidade nasal, dente
		lábios\cite{MTAGENTE}
	\end{itemize}
	\title{Lei de Bernoulli}
	
	\subsection{Aparelho Fonador}
	O estudo do aparelho fonador começa-se por suas estruturas e componentes importantes. Após um estudo detalhado dos fenômenos físicos e como se comportam é essencial também.

\section{Propriedades Físicas}

	\subsection{Lei Bernouli}
		Energia potencial e energica cinética em fluídos se mantém a mesma
		porém em proporções diferentes \cite{BradhStory}:
		
		\[
		P + \frac{\rho * v^2}{2} = Constante\\
		Sendo: \\
		\rho = Densidade do fluido \\
		P = Pressão no duto onde o fluído se encontra\\
		v =  velocidade da particula 
		\]
	\subsection{Critérios para Oscilação}
		Alguns critérios devem ser atendidos para que um determinador padrão de movimento seja considerado como uma oscilação mecânica, a saber:
		
		
		No sistema onde ocorre o movimento deve haver uma posição de equilíbrio estável, que é caracterizada por uma força restaurativa que sempre acelera o corpo em	movimento de volta para a sua posição de repouso.
		Deve haver inércia(no caso do sistema mecânico, a massa atua como propriedade de inércia) no sistema para superar esta posição de equilíbrio.
		A perda, em excesso, de energia por ciclo de oscilação deve ser zero.\ldots 
	
	\subsection{Tipos de Oscilação}
			De acordo com Titze~\cite{IngoTitze}, os tipos de oscilação são:
			\begin{itemize}
				\item  Oscilação Natural: Quando um sistema que se encaixa nos critérios anteriores se move sem interferência após um distúrbio inicial.
				\item Oscilação Natural: Quando um sistema que se encaixa nos critérios anteriores se move sem interferência após um distúrbio inicial.
				\item Oscilação Forçada: Requer uma fonte externa de condução que por si só é um
				oscilador. Dita grande parte do padrão de vibração do sistema.
				\item Oscilação Auto-Sustentável: Requer uma fonte de energia estável e uma interação não-linear entre os componentes internos ao sistema. As perdas de energia são compensadas, mantendo o padrão oscilatório.
			\end{itemize}
		
	\subsection{Tensao}
		Conceito de Tensão For por unidade de ar
		
		\[
		\sigma = \frac{F}{A}
		\]
		Sendo:
		f : força aplicada. \linebreak
		A: área de aplicação desta força.
		
	\subsection{Curva Força e Alongamento}
		Utiliza-se para não ser dependente da geometria do material. Utilizamos nas cordas vocais(?) por serem materias biológicos. Cria uma figura ilustrando comportamento da deformação das pregas vocais
		
		\includegraphics{figura1.png}
		
	\subsection{Viscosidade}
		É a velocidade de deformação(consequentemente, de restauração) de
		um determinado fluido quando atuam forças de tensão no mesmo. Matematicamente
		pode ser expresso conforme a equação seguinte: \citenum{Ingo}
		
		$
		\sigma = \eta * \frac{d*\epsilon}{dt}
		$
	
	\subsection{Reflexao de Som}
	
		Um fenômeno ligado a rigidez e amortecimento entre um meio e outro.\cite{MTAGENTE}
		Ondas quando tentam penetrar em um segundo meio, sendo o segundo meio rígido, as partículas primeiro meio se aglomeram tentando passar porém
		falham, seu acúmulo de particulas geram pressão que acabam criando uma outra onda no primeiro meio decorrente da primeira onda.\cite{HenryGray}
		
		O mesmo ocorre com o meio 2 sendo totalmente não rígido e o primeiro meio sendo bem rígido, Exaurindo excesso de particulas  do meio 1 no meio 2 criando rarefação no meio 1, o que cria uma outra onda de pressão negativa~\cite{FlanaganLandgraf}. A propagação é sempre em direção oposta à fonte, no caso é na direção contrária à coluna de ar(meio 1). 
		
	
	
		





\section{Sintetizadores de Voz}
	\subsection{Modelo Massa-Mola Auto Sustentável}:
	A fechadura e abertura da glote num sistema massa mola de apenas um lado

	\[
		P = (1 - \frac{a2}{a1})*(Ps- Pi) + Pi
	\]
		
	
	Versão simplificada da pressão massa mola
	\begin{itemize}
		\item P: Pressão resultante NA GLOTE
		\item a1: Areas de entrada da glote
		\item a2: Areas de saída  da glote
		\item Ps: Pressão subglotica
		\item Pi: Pressão sobre o trato vocal(Pressão input)
	\end{itemize}
	
		No modelo mono massa a1 = a2.No caso em que Pressão na GLOTE,P, seja  igual a pressão  supraglotal, indica que a inércia da coluna de ar acima da glote altera a pressão.
	
	
	\subsection{Synpath}
		Sintetizador computacional, criado por Lucero.Sintetizador concebido por Fraj \cite{JeanFrancis}
		Foi escolhido não usar o modelo multi-massa por variações não suaves  e é tematicamente muito complexo apesar instabilidades numéricas.
		Synpath utiliza como base de representação das pregas vocais o modelo mono massa. O trato vocal é representado por uma concatenação de tubos cilindricos os quais se propagam uma onda acustica.Superglótica não leva em conta o trato vocal.Perdas e viscosidade são considerada.
	
	\subsection{HMMs}
		Minera-se de várias partituras musicais para treino. Os dados minerados dessas músicas são fonemas, altura, intensidade e os intervalos, isto é relação com outras notas.Esses dados são convertidos e mapeados em "labels" dependentes de contexto~\cite{DegottexNada}.Após isso as HMM's são treinados através dos dados de treinamento usando o algoritmo EM.\cite{GudnasonNada}. Após isso ocorre a fase de sintesis, usa-se outra partitura para ser convertida em "labels" dependentes de contexto e estima-se quais "labels" preprocessadas são correspondentes.\cite{TakashiNose}
	
		\subsection{MHRSMM}
		Uma variação HSMM. Modelo de multipla regressão HSMM.
		Parametros importantes são μi e mi dos outputs pdfs
		
		\[ \mu_i = H_b_i \xi \] 
		\[	m_i   = H_p_i \xi  \]
		\[ \xi   = [1,v_1,v_2,...,v_L]^T \]	
		\[ \xi   = [1,v^T]^T	\]	
		
		Onde L é a dimensão do vetor de estilo e vi  é  a intensidade do  enésimo estilo de canto.
		
		Um exemplo de um vetor de estilos de canto de tamanho L =  2 e L = 3.
		
		\includegraphics{exemploHMM.png}
		
		\subsubsection{Controle do Sintentisador de voz cantada baseado em MRHSMM}
		
		\includegraphics{esquemaHMM.png}
		
		Durante a fase de síntese o usuário do programa adiciona vetores de estilos de acorodo com a intenção e a expressividade pretendida.
		Parametros de output como duração são gerados pelos vetores de estilos dados e matrizes de regressões treinadas usando MRHSMMs
		
		Resultado de todo esse processo é um sequência HSMM usando parametros de geração de fala
		
		MRHSMM possui uma dificuldade de gerar contorno F0 que acompanhe o contexto de mudança de altura das notas o author TAKASHI NOSE, propõe um treinamento de HSMM e HMM nos parametros 
		
	
	\subsection{FrameWorks Sintetisador de Voz}
	
		Frame Work de um Sistema Sintetisador de Voz:
		
		\includegraphics{frameWork.png}
		
		\subsubsection{Input}
		Consiste da partitura, letra e emoção.
		o input é analisado e derivado em uma transcrição fonética, alinhamento com a performance alvo ou dados contextuais.\cite{FrameWork}\linebreak
		
		\subsubsection{Expressão}
		Expressão músical é um conceito intuitivo porém dificil de se definir. A expressão é chave na percepção da qualidade e naturalidade musical.
		No caso da voz cantada implica-se usar vários outros parametro além de frequencia e amplitude. Psicológicamente contorno do timbre, vibrato, tremolo, timing fonético.\cite{FrameWork}\linebreak
	
	
	\subsection{Envoltoria F0}
	
	Envoltórias F0 são usadas para expressar informação linguistica, para-linguística e não-linguistica.\cite{SaitouF0}
	\linebreak
	
	As Envoltória F0 apresentam três (3) características importantes que fazem diferenciar uma voz falada a uma voz cantada.\cite{SaitouNada}
	\linebreak
	\begin{itemize}
		\item 1 - O alcance dinâmico de uma envoltória F0 é mais largo que o de uma voz falada
		\item 2 - A envoltória F0 corresponde e tende a se manter estável em uma nota. A mudança de nota de uma envoltória F0 corresponde a melodia da música
		\item 3 - Existem muita flutuações f0 que são apenas observadas em apenas vozes cantadas
	\end{itemize}
	
	
	\subsection{Sintese de Voz em Mandarím}
		Utiliza-se a técnica HNM para a sintese da voz cantada em mandarím. HNM significa , "harmonic plus noise model".
		O modelo HNM divide o espectro de um sinal em dois(2) com larguras não iguais para modelagem melhor do espectro.\cite{LinRobos}
	

 

\section{Sistema Auditivo}
	\subsection{Introdução}
	O Sistema auditivo consiste em componentes periféricos e centrais.Atualmente a maior parte do conhecimento do funcionamento dos sistemas auditivos deriva de estudos de animais não humanos.\cite{Foundation1} 
	
	Sistema auditivo diferencia-se entre espécies em jeitos interessantes.
	Por exemplo algumas espécies tem caracteristicas diferentes relacionadas aos sinais vocais mais utilizados por ela mesma.\cite{Foundation1}
	
	\subsection{Intensidade}
	Como a frequência de um estímulo a intensidade dele é processado e codificado sub-cortical nos dois lados do cérebro em todos os níveis no cerébro.\cite{Foundation1}
		


\section{Voz e Propriedades Linguísticas}
	Uma divisão importante de acordo com Flanagen, são as letras separados em classificações Vogais e Consoantes que se associam a um movimento do trato vocal correspondente~\cite{JFlanagan}.
	
	\includegraphics{tabelaConsoantes.png}
	
	\subsection{Vogais}
	
	O trato vocal ao produzir uma vocal, em um articação normal, mantém-se relativamente estável.Há uma opção de contribuição das cavidades nasais, uma cobertura porém é negligenciável.Baseado nessas caracteristicas é divididos todas as consonantes. A tabela abaixo explica:
	
	
	\subsection{Consoantes}
	Sons produzidos com constrições em algum ponto no trato vocal. Dividido em quatro (4) classes, baseados em duas funcionalidades binárias, sonorant e continuant.
	
	
		\subsubsection{Sonorant}	 
			Sonorant pode ser traduzido como "cantado". Consoantes Sonorant são sons que não aumentam a pressão do ar dentro do trato vocal pois a constrição não é muito justa ou o palato continua aberto, deixando ar escapar por ele.
			
		\subsubsection{Continuant}
			Uma consoante discontinuant é produzida por um fechamento completo em algum ponto no trato vocal.

