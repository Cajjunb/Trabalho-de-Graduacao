
\chapter{Aparelho Fonador}
\section{Anatomia da Voz}
	Para estudar a produção e a síntese da voz, é necessário ter um conhecimento acerca da anatomia e do funcionamento físico da voz \citenum{Foundation1}. Sendo assim, as subseções seguintes descreverão brevemente detalhes da anatomia do sistema fonador humano e como o som é produzido, moldado e influenciado por este sistema.
	
	\begin{figure}
		\centering
		\includegraphics[scale=0.5]{aparelhoFonador}
		\caption{Aparelho Fonador}
		\label{fig:aparelhoFonador}
	\end{figure}
	\subsection{Aparelho Fonador}
	O estudo do aparelho fonador começa-se por suas estruturas e componentes importantes. Após um estudo detalhado dos fenômenos físicos e como se comportam é essencial também.
		
	A Figura \ref{fig:aparelhoFonador} \citenum{Foundation1}, mostra os órgãos associados com a produção da voz. 
	
	Dentro das condições normais, a voz é produzida quando um fluxo de ar vindo dos pulmões é convertido em energia acústica através da vibração das pregas vocais, localizadas na laringe. Os padrões de vibrações resultantes são moldados acusticamente quando o som passa pelo trato vocal acima da laringe. O sistema respiratório serve como uma
	fonte de potência para a produção do som, sendo responsável por movimentar o ar através do trato vocal. A laringe atua como um oscilador convertendo a potência aerodinâmica produzida em energia sonora, sendo frequentemente retratada como a fonte da voz. No entanto, a mais importante função da laringe não é a produção de som, e sim, vedar as vias aéreas aos pulmões completamente, protegendo-as de objetos estranhos ou líquidos, principalmente durante a deglutinação. De maneira análoga, a laringe serve como uma válvula de acesso às vias respiratórias e por essa característica, atua também no controle do fluxo de ar que por elas passam. Sendo assim,é fácil notar que há uma necessidade de mobilidade para toda estruturada laringe, logo é de se esperar que sua estrutura seja formada em sua maioria por cartilagens. De fato o é, com exceção de um osso chamado de Hioide, a laringe é basicamente formada por cartilagens e músculos. A seguir, analisaremos brevemente a dinâmica dos músculos e cartilagens da laringe.
	
\subsection{Músculos e Cartilagens}
Os músculos e cartilagens atuam diretamente no processo de abdução e adução das pregas vocais. Estas estão localizadas dentro da laringe e devido à dinâmica das cartilagens e dos músculos, podem executar os movimentos citados de forma a produzir som.

\subsubsection{Cartilagens da Laringe }
\begin{figure}
	\centering
	\includegraphics[scale=0.5]{musculosCartilagens}
	\caption{Secção coronal da laringe e parte superior da traquéia}
	\label{fig:Cartilagens}
\end{figure}
A Figura \ref{fig:Cartilagens}, mostra uma secção da laringe, detalhando as cartilagens presentes.
De maneira sucinta, estas cartilagens servem como base de interconexão para os músculos intrínsecos ao redor da laringe. Dentre as cartilagens acima, a epiglote é responsável por vedar as vias respiratórias movimentando-se sobre a entrada das mesmas. O resto das cartilagens garantem a mobilidade da laringe em conjunto com outras estruturas como por exemplo o sternum.


\subsubsection{Músculos da Laringe}
Os músculos na laringe podem ser divididos em dois grupos, os intrínsecos e os extrínsecos \citealp{Giovanni20041}. Os músculos intrínsecos interconectam as cartilagens da laringe, ao passo que, os extrínsecos conectam a laringe à outras estruturas externas, como o osso hióide. A Figura \ref*{fig:musculos} detalha alguns dos músculos intrínsecos da laringe. Alguns desses músculos têm influência direta em algumas características da voz. Por exemplo, o músculo cricotiroideo é o músculo primário utilizado no controle do tom da voz. Por sua vez, o músculo cricoaritenoideo posterior atua na abdução das pregas vocais, ao passo que o músculo interaritenoideo atua como adutor das pregas vocais.

\begin{figure}
	\centering
	\includegraphics[scale=0.5]{musculosLaringe}
	\caption{Músculos Intrínsecos da Laringe}
	\label{fig:musculos}
\end{figure}

Os músculos extrínsecos, Figura \ref{fig:musculoExterno}, atuam basicamente no movimento da laringe, agindo como depressor e elevador da estrutura laríngea. Além disso também conectam estruturas do trato vocal à estrutura laríngea, como por exemplo a língua ao osso hioide.


\begin{figure}
	\centering
	\includegraphics[scale=0.5]{musculosExternosLaringe}
	\caption{Músculos Extrínsecos da Laringe }
	\label{fig:musculoExterno}
\end{figure}

\subsection{Pregas Vocais}

As pregas vocais, como dito anteriormente, estão localizadas dentro da laringe, mais específicamente na parte superior da traqueia. Elas estão posteriormente ligadas às cartilagens aritenoides, e anteriormente ligadas à cartilagem tireoide. As suas bordas exteriores estão ligadas a músculos na laringe, enquanto as suas bordas interiores são livres.

As bordas das pregas vocais são construídas de epitélio, sendo compostas também de algumas fibras musculares. As pregas vocais são bandas triangulares planas de cor branca e acima de ambos os lados destas, se encontram as pregas vestibulares ou falsas pregas vocais. 

O espaço entre as pregas vocais é chamado de glote, sendo que o que está acima da glote é denominado supraglotal e o que está abaixo é denominado subglotal. A Figura 1.5 mostra em mais detalhes a anatomia das pregas vocais, os componentes musculares e as cartilagens atuantes.

\begin{figure}
	\centering
	\includegraphics[scale=0.5]{cordasVocais}
	\caption{ Cordas Vocais e Componentes}
	\label{fig:cordasVocasi}
\end{figure}

\section{Fundamentos Biofísicos para a Produção da Voz}
	\subsection{ A Biomecanica da Laringe}
	Primeiramente, devemos ter em mente que o principal papel da laringe não é a produção de voz e sim a proteção das vias respiratórias. Dito isto, podemos fazer uma simples analise, visto que apesar de seu papel principal, a laringe também atua como um instrumento da fala humana. Se analisarmos os instrumentos criados pelo homem podemos notar que estes dependem basicamente de sua geometria, do material que o compõe e da interação de suas partes acústicas. Do mesmo modo, a laringe possui uma determinada geometria, é composta por tecido humano e em conjunto como trato vocal compõe a parte acústica do nosso corpo. Entretanto, nada é tão simples, a sua geometria e as propriedades do material humano envolvidos na produção do som são bastante irregulares\cite{IngoTitze}. 
	
	Outra analogia interessante é sobre o instrumento e quem o utiliza. Um bom pianista por exemplo, sua musica é boa porque ele é habilidoso com o instrumento? Ou sua musica é boa por que o instrumento é bem feito e o som gerado por este é agradável? Ou os dois? Essas perguntas também podem ser feitas com respeito a voz. Para entendermos o que influência na qualidade da síntese da voz é necessário analisar a biomecânica da voz, que nada mais é analisar o movimento do material vivo e as forças atuantes sobre ele\cite{IngoTitze}.
	
	\subsubsection{ Fatores Biológicos que Afetam a Produção de Som na Laringe }
	
	A parte da (bio)mecânica que se relaciona diretamente com a atuação da laringe na produção do som é a mecânica dos meios contínuos, que é a parte da mecânica que lida com a matéria distribuída sobre uma determinada região no espaço, e consequentemente, se contrapõe à mecânica de partículas. Dentro da mecânica de meios contínuos, mais especificamente, a parte que irá nos auxiliar no estudo do comportamento da laringe se chama mecânica de sólidos e fluídos.
	
	Dito isto, analisaremos a seguir alguns conceitos físicos que tem forte ligação com os processos que ocorrem na região da laringe durante a produção do som: 

	\subsection{Tensao}
	Tensão é quantidade de força por unidade de área~\cite{IngoTitze} , podemos escrever na forma da equação

	\[
	\sigma = \frac{F}{A}
	\]
	Sendo:
	f : força aplicada. \linebreak
	A: área de aplicação desta força.
	
	\subsection{Curva Força e Alongamento}
	Utiliza-se para não ser dependente da geometria do material. Utilizamos nas cordas vocais(?) por serem materias biológicos. Cria uma figura ilustrando comportamento da deformação das pregas vocais
	
	\includegraphics{figura1.png}
	
	\subsection{Viscosidade}
	É a velocidade de deformação(consequentemente, de restauração) de
	um determinado fluido quando atuam forças de tensão no mesmo. Matematicamente
	pode ser expresso conforme a equação seguinte: \citenum{Ingo}
	
	$
	\sigma = \eta * \frac{d*\epsilon}{dt}
	$
	
	
	\subsection{Reflexao de Som}
	
	Um fenômeno ligado a rigidez e amortecimento entre um meio e outro.\cite{MTAGENTE}
	Ondas quando tentam penetrar em um segundo meio, sendo o segundo meio rígido, as partículas primeiro meio se aglomeram tentando passar porém
	falham, seu acúmulo de particulas geram pressão que acabam criando uma outra onda no primeiro meio decorrente da primeira onda.\cite{HenryGray}
	
	O mesmo ocorre com o meio 2 sendo totalmente não rígido e o primeiro meio sendo bem rígido, Exaurindo excesso de particulas  do meio 1 no meio 2 criando rarefação no meio 1, o que cria uma outra onda de pressão negativa~\cite{FlanaganLandgraf}. A propagação é sempre em direção oposta à fonte, no caso é na direção contrária à coluna de ar(meio 1). 
	
	
	\subsection{Fluxo de Ar na Glote}
	
	Fluxo de ar na Glote:
	
	Como descrito no artigo de Elias temos informações como descobrimos a pressão via fluxo de ar\cite{eliasamadeudesouza}
	
	\[
	U_g = +-(\frac{-a_m}{A_*}+[(a_m)^2 +- (\frac{4K_t}{C^2\rho})](P^+_s - P^-_i))
	\]
	
	\[
	$		A^*  = \text{Área efetiva computada pelas areas A_i e A_s}$
	\]
	
	\[
	$		\rho = \text{Densidade do Ar}$
	\]
	\[
	$		c = \text{Velocidade do som}$
	\]
	
	\[
	$		P_s e P_i = \text{Pressão de incidencia na entra e saída da glote}$
	\]
	
	Uma vez descoberto o fluxo, as pressões de reflexão P^s_ e P^+_i podem ser encontradas com a seguinte equação:
	
	\[
	P^-_s = P^+_s - (\rho c / A_s) U_g
	\]
	
	
	\[
	P^+_i = P^-_i - (\rho c / A_i) U_g
	\]
	
	Quando ocorre o fechamento da glote então alguns parametros assumem valores conhecidos. a(t) = 0 , 
	$P_g = \frac{P_s-P_i}{2}$ e $U_g = 0$	
	
	
	



\section{Propriedades Físicas}

	\subsection{Lei Bernouli}
		Energia potencial e energica cinética em fluídos se mantém a mesma
		porém em proporções diferentes \cite{BradhStory}:
		
		\[
		P + \frac{\rho * v^2}{2} = Constante\\
		Sendo: \\
		\rho = Densidade do fluido \\
		P = Pressão no duto onde o fluído se encontra\\
		v =  velocidade da particula 
		\]
	\subsection{Critérios para Oscilação}
		Alguns critérios devem ser atendidos para que um determinador padrão de movimento seja considerado como uma oscilação mecânica, a saber:
		
		
		No sistema onde ocorre o movimento deve haver uma posição de equilíbrio estável, que é caracterizada por uma força restaurativa que sempre acelera o corpo em	movimento de volta para a sua posição de repouso.
		Deve haver inércia(no caso do sistema mecânico, a massa atua como propriedade de inércia) no sistema para superar esta posição de equilíbrio.
		A perda, em excesso, de energia por ciclo de oscilação deve ser zero.\ldots 
	
	\subsection{Tipos de Oscilação}
			De acordo com Titze~\cite{IngoTitze}, os tipos de oscilação são:
			\begin{itemize}
				\item  Oscilação Natural: Quando um sistema que se encaixa nos critérios anteriores se move sem interferência após um distúrbio inicial.
				\item Oscilação Natural: Quando um sistema que se encaixa nos critérios anteriores se move sem interferência após um distúrbio inicial.
				\item Oscilação Forçada: Requer uma fonte externa de condução que por si só é um
				oscilador. Dita grande parte do padrão de vibração do sistema.
				\item Oscilação Auto-Sustentável: Requer uma fonte de energia estável e uma interação não-linear entre os componentes internos ao sistema. As perdas de energia são compensadas, mantendo o padrão oscilatório.
			\end{itemize}
		






	\section{Sistema Auditivo}
	\subsection{Introdução}
	O Sistema auditivo consiste em componentes periféricos e centrais.Atualmente a maior parte do conhecimento do funcionamento dos sistemas auditivos deriva de estudos de animais não humanos.\cite{Foundation1} 
	
	Sistema auditivo diferencia-se entre espécies em jeitos interessantes.
	Por exemplo algumas espécies tem caracteristicas diferentes relacionadas aos sinais vocais mais utilizados por ela mesma.\cite{Foundation1}
	
	\subsection{Intensidade}
	Como a frequência de um estímulo a intensidade dele é processado e codificado sub-cortical nos dois lados do cérebro em todos os níveis no cerébro.\cite{Foundation1}
	
	
	
	\section{Voz e Propriedades Linguísticas}
	Uma divisão importante de acordo com Flanagen, são as letras separados em classificações Vogais e Consoantes que se associam a um movimento do trato vocal correspondente~\cite{JFlanagan}.
	
	\includegraphics{tabelaConsoantes.png}
	
	\subsection{Vogais}
	
	O trato vocal ao produzir uma vocal, em um articação normal, mantém-se relativamente estável.Há uma opção de contribuição das cavidades nasais, uma cobertura porém é negligenciável.Baseado nessas caracteristicas é divididos todas as consonantes. A tabela abaixo explica:
	
	
	\subsection{Consoantes}
	Sons produzidos com constrições em algum ponto no trato vocal. Dividido em quatro (4) classes, baseados em duas funcionalidades binárias, sonorant e continuant.
	
	
	\subsubsection{Sonorant}	 
	Sonorant pode ser traduzido como "cantado". Consoantes Sonorant são sons que não aumentam a pressão do ar dentro do trato vocal pois a constrição não é muito justa ou o palato continua aberto, deixando ar escapar por ele.
	
	\subsubsection{Continuant}
	Uma consoante discontinuant é produzida por um fechamento completo em algum ponto no trato vocal.
	

 

